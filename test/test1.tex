\documentclass[oneside, a4paper]{article}
\usepackage[a4paper,width=150mm,top=25mm,bottom=25mm,bindingoffset=6mm]{geometry}
\usepackage{xfrac}
\usepackage{amsmath}

\newenvironment{program}[1]
{%
    \newcommand{\print}[1]{##1}
    #1%
}
{}

\usepackage{xfrac}
\usepackage{amsmath}
\author{Julian Karrer}
\title{klarTeXt Test File}
\begin{document}
\maketitle

\section*{Arithmetic}

\subsection*{Integrals}
\begin{program}
\begin{align*}
    \print{\int_{-5}^5 \Theta(x) \, dx}\\
    \print{\int_0^\pi \sin(x) \, dx}\\
    \print{\int_0^\pi\int_{0}^{\pi} y\sin(x) \, dx\, dy}\\
    \print{\int_1^3\int_2^4 9x^3y^2 \, dy\, dx}\\
    \print{\int_0^1\int_{x^2}^x x+3 \, dy\, dx}\\
    \print{\frac{7}{12}}\\
\end{align*}
\end{program}

\subsection*{Sums}
\begin{program}
\begin{align*}
    \print{\sum_{i=1}^5 \exp(i)}\\
    \print{\sum_{i=1}^{10} \ln(i)}\\
    \print{\ln(10!)}\\
    \print{\sum_{i=0}^{3}\cos\left(\frac{\pi i}{2}\right)}\\
    \print{\sum_{i=1}^{4} i\sin\left(\frac{\pi}{i+1}\right)}\\
    \print{\sum_{i=1}^{3} \log(i+e)}\\
    \print{\sum_{i=0}^{4} \frac{\pi^i}{i!}}\\
    \print{\sum_{i=1}^{5} \Theta(i-3)}\\
\end{align*}
\end{program}

\subsection*{Products}
\begin{program}
\begin{align*}
    \print{\prod_{i=1}^{10} i}\\
    \print{10!}\\
    \print{\prod_{i=1}^4 \exp(i)}\\
    \print{\prod_{i=0}^{3}\cos\left(\frac{\pi i}{2}\right)}\\
    \print{\prod_{i=1}^{4} \sin\left(\frac{\pi}{i+1}\right)}\\
    \print{\prod_{i=1}^{3} \ln(i+e)}\\
    \print{\prod_{i=0}^{3} \frac{\pi^i}{i!}}
\end{align*}
\end{program}


\subsection*{Infinity and Beyond}
\begin{program}
\begin{align*}
    \print{-5\cdot \infty}\\
    \print{\frac{5}{0}}
\end{align*}
\end{program}

\subsection*{Special Functions}
\begin{program}
    \begin{align*}
        \print{\Gamma(4+1)}\\
        \print{\Phi(0)}\\
        \print{\arcsin\left(\sfrac{1}{2}\right)}\\
    \end{align*}
\end{program}


\subsection*{Random Stuff}
\begin{program}
\begin{align*}
    \print{e(2+1)}\\
    \text{Noah}(f,g,x,y) = f(g(x, y), y)\\
    f(x,y) = y^2\cdot \sin{\frac{x}{y}}\\
    \print{f}\\
    g_2(x,y) = 1\\
    \print{\text{Noah}(f,g_2,1,1)}\\
    \print{f(1,1)}\\
    \text{Laura}(x,y) = e^{-g_2(x,y)}\\
    \print{\frac{\text{Laura}(\pi, e)}{\Gamma(3)}\cdot 2!}\\
    a = b\\
    b = c\\
    c = 4\\
    g(x) = 2\\
    \text{comp}(f,g,x) = f(g(x))\\
    \print{\text{comp}(\sin,\cos,\pi)}\\
    m = \min\\
    \print{m (3,2)}\\
    \print{m (3)}\\
    \print{\min ()}\\
    \print{\max ()}\\
    \print{m\left(1,2,3,4,45,65,7,\frac{1}{2}\right)}\\
    \print{\Theta(0)}\\
\end{align*}
\end{program}



\end{document}