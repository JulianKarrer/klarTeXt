\documentclass[oneside, a4paper]{article}
\usepackage[a4paper,width=150mm,top=25mm,bottom=25mm,bindingoffset=6mm]{geometry}
\usepackage{xfrac}
\usepackage{amsmath}

\usepackage{xfrac}
\usepackage{amsmath}

\newcounter{klarCall}
\newcommand{\callKlarTeXt}{%
  \immediate\write18{%
    ../target/release/klarTeXt "$(realpath \jobname.tex)" true%
  }%
  \checkKlarTeXtError%
  \checkKlarTeXtWarnings%
}

\newcommand{\checkKlarTeXtError}{%
  \IfFileExists{klarTeXt_\jobname_error.txt}{%
    \begingroup
      \newread\errfile
      \openin\errfile=klarTeXt_\jobname_error.txt
      \def\processline{%
        \read\errfile to \temp
        \ifeof\errfile
          \let\next\relax
        \else
          \edef\templine{\expandafter\detokenize\expandafter{\temp}}%
          \begingroup
          \catcode`\ =12 % treat spaces as literals
          \GenericError{}{\templine}{}{}%
          \endgroup
          \let\next\processline
        \fi
        \next
      }%
      \processline
      \closein\errfile
    \endgroup
  }{}%
}


\newcommand{\checkKlarTeXtWarnings}{%
  \IfFileExists{klarTeXt_\jobname_warning.txt}{%
    \begingroup
      \newread\errfile
      \openin\errfile=klarTeXt_\jobname_warning.txt
      \def\processline{%
        \read\errfile to \temp
        \ifeof\errfile
          \let\next\relax
        \else
          \edef\templine{\expandafter\detokenize\expandafter{\temp}}%
          \begingroup
          \catcode`\ =12 % treat spaces as literals
          \PackageWarning{klarTeXt}{\templine}{}{}%
          \endgroup
          \let\next\processline
        \fi
        \next
      }%
      \processline
      \closein\errfile
    \endgroup
  }{}%
}


\AtBeginDocument{\callKlarTeXt}

\newcommand{\ddx}[1]{\frac{\partial}{\partial #1}}

\newenvironment{program}[1][]{%
  \newcommand{\print}[1]{%
    ##1 = %
    \stepcounter{klarCall}%
    \edef\tempFile{klarTeXt_\jobname_\theklarCall.tex}%
    \IfFileExists{\tempFile}{%
      \input{\tempFile}%
    }{%
      \text{unknown}%
    }%
  }%
  \newcommand{\simplify}[1]{%
    ##1 = %
    \stepcounter{klarCall}%
    \edef\tempFile{klarTeXt_\jobname_\theklarCall.tex}%
    \IfFileExists{\tempFile}{%
      \input{\tempFile}%
    }{%
      \text{unknown}%
    }%
  }%
}{}
\usepackage{xfrac}
\usepackage{amsmath}
\author{Julian Karrer}
\title{klarTeXt Test File}
\begin{document}
\maketitle

\section*{Arithmetic}

\subsection*{Test}
\begin{program}
    \begin{align*}
        \simplify{\int_{0}^{2\pi} {\ddx{x}\sin(x)} + \sin(x) + 5 \,dx} \\
        \simplify{\int_{-e}^{2\pi} (\sum_{i=2}^{5} \frac{x}{i})\,dx}   \\
        \simplify{\int_{0}^{\pi} \Phi(x) \,dx}                         \\
        \simplify{\int_{-1}^{2} \frac{1}{x} \,dx}                      \\
        % \simplify{\int_{-1}^{5} \exp(x) + 2xy \,dx}                    \\
        % \simplify{\int_{-1}^{5} 2xy  \,dx}                             \\
        % f(x,i) = \int_{i}^{2i} \arccos(x^t)\,dt \\
        % \print{f\left(\frac{1}{2}, 5\right)}\\
        % \simplify{\ddx{x} f}\\
        % \simplify{2xy + x\cdot 5y}\\
        % \simplify{x \cdot x^3 x x x x x x + y y^5}\\
        % \simplify{x^2 x^5}\\
        % \simplify{1 + 2x^8 + 5\cdot 2+ 5 \cdot 3 + 5 \cdot 4}\\
        % \simplify{5xy + 5xy}\\
        % \simplify{2x + x + 5xy + \frac{0}{5} + xy \cdot xy + 3x \cdot 4x}\\
        % \simplify{15\cdot\frac{0}{xyz}}
        % f(x,y,\rho) = \frac{1}{2}(x-1)^2 + \frac{1}{2}y^2+\rho\frac{1}{2}(y-\cos(x))^2\\
        % \simplify{\ddx{x} f}\\
        % \simplify{\ddx{y} f}\\
        % \simplify{\ddx{x}{\ddx{x} f}}\\
        % \simplify{\ddx{y}{\ddx{y} f}}\\
        % \simplify{\ddx{y}{\ddx{x} f}}\\
        % \simplify{\ddx{x}{\ddx{y} f}}\\
    \end{align*}
\end{program}



\newpage
\subsection*{Symbolic integration}
Integrals are first attempted to be symbolically integrated - if symbolic integration fails, definite integrals are numerically evaluated using Gauss-Kronrod G7K15 quadrature.
The following set of rules is used repeatedly and recursively:
\begin{align*}
    \int n \,dx                     & = x n + C                                                    & \text{constants}                           \\
    \int -f(x) \,dx                 & = - \int f(x) \,dx                                           & \text{negatives}                           \\
    \int f(x) + g(x) \,dx           & = \int f(x) \,dx  + \int g(x) \,dx                           & \text{sums}                                \\
    \int \sum_{i=a}^{b} f(x,i) \,dx & = \sum_{i=a}^{b} \int f(x,i) \,dx                            & \text{$\Sigma$-sums}                       \\
    \int \sum_{x=a}^{b} f(x) \,dx   & = x \left(\sum_{i=a}^{b} \int f(x)\right) +C                 & \text{shadowing $\Sigma$-sums}             \\
    \int \prod_{x=a}^{b} f(x) \,dx  & = x \left(\prod_{x=a}^{b} \int f(x)\right) +C                & \text{shadowing $\prod$-products}          \\
    \int \frac{f(x)}{g(y)} \,dx     & = \frac{1}{g(y)} \int f(x) \,dx                              & \text{fractions}                           \\
    \int \frac{n}{x} \,dx           & = n \ln(x)     +C                                            & \text{}                                    \\
    \int x^{n} \,dx                 & = \frac{x^{n+1}}{n+1}                                        & \text{if $n\neq -1$}                       \\
    \int \sqrt[n]{x} \,dx           & =  \frac{n}{n+1} x^{\frac{n+1}{n}}                           & \text{if $n\neq -1$}                       \\
    \int \sin(x) \,dx               & = -\cos(x)                                       + C         & \text{trigonometric functions}             \\
    \int \cos(x) \,dx               & = \sin(x)                                        + C         & \text{}                                    \\
    \int \tan(x) \,dx               & = -\frac{\ln\left(\cos(x)^2\right)}{2}           + C         & \text{}                                    \\
    \int \arcsin(x) \,dx            & = x \arcsin(x) + \sqrt{1 - x^2}                  + C         & \text{inverse trigonometric functions}     \\
    \int \arccos(x) \,dx            & = x \arccos(x) - \sqrt{1 - x^2}                  + C         & \text{}                                    \\
    \int \arctan(x) \,dx            & = x \arctan(x) - \frac{\ln\left(x^2+1\right)}{2} + C         & \text{}                                    \\
    \int \sinh(x) \,dx              & = \cosh(x)                                       + C         & \text{hyperbolic functions}                \\
    \int \cosh(x) \,dx              & = \sinh(x)                                       + C         & \text{}                                    \\
    \int \tanh(x) \,dx              & = \ln\left(\cosh(x)\right)                       + C         & \text{}                                    \\
    \int \ln(x) \,dx                & = x \left(\ln(x) - 1\right)                      + C         & \text{logarithms}                          \\
    \int \lg(x) \,dx                & = \frac{x}{\ln(10)} \left(\ln(x) - 1\right)      + C         & \text{}                                    \\
    \int \exp(x) \,dx               & = \exp(x)                                        + C         & \text{exponentials}                        \\
    \int \Theta(x) \,dx             & = x \Theta(x)                                    + C         & \text{Heaviside Theta}                     \\
    \int \Phi(x) \,dx               & = x \Phi(x) + \frac{1}{\sqrt{2\pi}} \exp(-\frac{x^2}{2}) + C & \text{CDF of standard normal distribution} \\
\end{align*}
where n is constant with respect to x. If n cannot statically be proven to be different from -1, since for example it occurs open in a term of a function body, some rules cannot be applied.


\subsection*{Simplicification}
The following rules for simplification of terms, including open terms with free variables, are implemented. Where the symmetric case of a rule applies due to commutativity, this should also be implemented.
\begin{align*}
    x \cdot 0 \cdot y              & = 0                                  & \text{absorbing elements}            \\
    x + 0                          & = x                                  & \text{neutral elements}              \\
    x \cdot 1                      & = x                                  &                                      \\
    \frac{x}{1}                    & = x                                  &                                      \\
    \sqrt[1]{x}                    & = x                                  &                                      \\
    x^1                            & = x                                  &                                      \\
    ax + bx = (a+b)x               & = x                                  & \text{combining like terms}          \\
    x^a + x^b = x^{a+b}            & = x                                  &                                      \\
    \forall a>b:\, \sum_{a}^{b} x  & = 0                                  & \text{empty sums and products}       \\
    \forall a>b:\, \prod_{a}^{b} x & = 1                                  &                                      \\
    -(-x)                          & = x                                  & \text{operations that cancel}        \\
    \sqrt[y]{x^y}                  & = x                                  & \textit{roots and powers}            \\
    \left(\sqrt[y]{x}\right)^y     & = x                                  &                                      \\
    x + (-x)                       & = 0                                  & \textit{plus and minus}              \\
    (-x) + x                       & = 0                                  &                                      \\
    x \cdot \frac{a}{x}            & = a                                  & \textit{multiplication and division} \\
    -5x \cdot 2y \cdot (-6i)       & = 5x \cdot 2y \cdot 6i               & \text{parity of minus in products}   \\
    -5x \cdot (-2y) \cdot (-6i)    & = -\left(5x \cdot 2y \cdot 6i\right) & \textit{even or odd number of $-$}   \\
    x^0                            & = 1                                  & \text{trivial due to argument}       \\
    \frac{x}{x}                    & = 1                                  &                                      \\
    \frac{0}{x}                    & = 0                                  &                                      \\
\end{align*}
The above rules for analytically solvable integrals and constant folding (if subexpressions are values, evaluate the entire expression, e.g. `5 + 7' becomes `12' while `5x' remains `5x') are also implemented.

% \subsection*{Integrals}

% \begin{program}
%     \begin{align*}
%         \print{\int_{-5}^5 \Theta(x) \, dx}\\
%         \print{\int_0^\pi \sin(x) \, dx}\\
%         \print{\int_0^\pi\int_{0}^{\pi} y\sin(x) \, dx\, dy}\\
%         \print{\int_1^3\int_2^4 9x^3y^2 \, dy\, dx}\\
%         \print{\int_0^1\int_{x^2}^x x+3 \, dy\, dx}\\
%         \print{\frac{7}{12}}\\
%     \end{align*}
% \end{program}

% \subsection*{Sums}
% \begin{program}
% \begin{align*}
%     \print{\sum_{i=1}^5 \exp(i)}\\
%     \print{\sum_{i=1}^{10} \ln(i)}\\
%     \print{\ln(10!)}\\
%     \print{\sum_{i=0}^{3}\cos\left(\frac{\pi i}{2}\right)}\\
%     \print{\sum_{i=1}^{4} i\sin\left(\frac{\pi}{i+1}\right)}\\
%     \print{\sum_{i=1}^{3} \log(i+e)}\\
%     \print{\sum_{i=0}^{4} \frac{\pi^i}{i!}}\\
%     \print{\sum_{i=1}^{5} \Theta(i-3)}\\
% \end{align*}
% \end{program}

% \subsection*{Products}
% \begin{program}
% \begin{align*}
%     \print{\prod_{i=1}^{10} i}\\
%     \print{10!}\\
%     \print{\prod_{i=1}^4 \exp(i)}\\
%     \print{\prod_{i=0}^{3}\cos\left(\frac{\pi i}{2}\right)}\\
%     \print{\prod_{i=1}^{4} \sin\left(\frac{\pi}{i+1}\right)}\\
%     \print{\prod_{i=1}^{3} \ln(i+e)}\\
%     \print{\prod_{i=0}^{3} \frac{\pi^i}{i!}}
% \end{align*}
% \end{program}


% \subsection*{Infinity and Beyond}
% \begin{program}
% \begin{align*}
%     \print{-5\cdot \infty}\\
%     \print{\frac{5}{0}}
% \end{align*}
% \end{program}

% \subsection*{Special Functions}
% \begin{program}
%     \begin{align*}
%         \print{\Gamma(4+1)}\\
%         \print{\Phi(0)}\\
%         \print{\arcsin\left(\sfrac{1}{2}\right)}\\
%     \end{align*}
% \end{program}


% \subsection*{Random Stuff}
% \begin{program}
% \begin{align*}
%     \print{e(2+1)}\\
%     \text{Noah}(f,g,x,y) = f(g(x, y), y)\\
%     f(x,y) = y^2\cdot \sin{\frac{x}{y}}\\
%     \print{f}\\
%     g_2(x,y) = 1\\
%     \print{\text{Noah}(f,g_2,1,1)}\\
%     \print{f(1,1)}\\
%     \text{Laura}(x,y) = e^{-g_2(x,y)}\\
%     \print{\frac{\text{Laura}(\pi, e)}{\Gamma(3)}\cdot 2!}\\
%     a = b\\
%     b = c\\
%     c = 4\\
%     g(x) = 2\\
%     \text{comp}(f,g,x) = f(g(x))\\
%     \print{\text{comp}(\sin,\cos,\pi)}\\
%     m = \min\\
%     \print{m (3,2)}\\
%     \print{m (3)}\\
%     \print{\min ()}\\
%     \print{\max ()}\\
%     \print{m\left(1,2,3,4,45,65,7,\frac{1}{2}\right)}\\
%     \print{\Theta(0)}\\
% \end{align*}
% \end{program}



\end{document}